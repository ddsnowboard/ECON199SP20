\documentclass[letterpaper]{article}
\usepackage{amsmath}
\usepackage{algorithm}
\usepackage{multirow}
\usepackage{graphicx}

\DeclareMathOperator*{\argmin}{arg\,min}

\begin{document}
\title{ECON 199 Problem Set 2}
\author{Will Koster (jameswk2; 651028726) \and Javier Garza (javierg2; 667146159)}
\date{Feb 29, 2020}
\maketitle

\clearpage

\section{Problem 1}
\begin{enumerate}
    \item The outcomes for each player are listed below, ranked from best to worst:
        \begin{enumerate}
            \item I cheat and the other person studies (since I get a good grade without doing any work)
            \item I study and the other person studies (since I get a good grade, although I had to work to get it)
            \item I study and the other person cheats off me (I still get a good grade from doing work)
            \item We both cheat and get caught
        \end{enumerate}
        Some would argue that the middle two are properly ordered, since it's somewhat unjust that the other player gets the same grade from doing no work, but they are the same in terms of strict utility, so they should get the same score in the game. \\
\begin{tabular}{|l|l|l|l|}
\multicolumn{4}{c}{P2}                      \\ \hline
\multirow{3}{*}{P1} &   & Cheat        & Honest        \\ \hline
                       & Cheat & $0,0$ & $\overline2, \overline1$ \\ \hline
                       & Honest & $\overline1, \overline2$ & $1,1$ \\ \hline
\end{tabular} \\
\item The two Nash equilibria come when one player cheats and the other is honest.
\item \begin{enumerate}
        \item This would likely not change the game at all. It would create a focal point at (Honest, Honest), but the belief that your opponent is more likely to choose Honest just increases the expected utility of cheating. It could possibly cause guilt for cheating even if the cheater is not caught, which could change the outcome only in cases where the guilt outweighs the work required to be honest (id est, in easy classes).
        \item Assuming the only way to get caught is if both players try to cheat, whether the payoff for getting caught is $0$ or $-1,000,000$ won't change the outcome. Since any value less than $1$ can be unilaterally improved upon by either player, decreasing the payoff won't alter the Nash equilibria.
        \item If both players cheat but they don't get caught, both players still didn't study. Even if they copy each other, they won't have the right answers. This means that, although the payoff for both players cheating might increase slightly, it will still be worse than if either of them were honest, which means the final outcome won't change.
    \end{enumerate}
\end{enumerate}
\end{document}
